% \iffalse meta-comment
% vim: textwidth=75
%<*internal>
\iffalse
%</internal>
%<*internal>
\fi
\def\nameofplainTeX{plain}
\ifx\fmtname\nameofplainTeX\else
  \expandafter\begingroup
\fi
%</internal>
%<*install>
\input docstrip.tex
\keepsilent
\askforoverwritefalse
\preamble
-----------------------:| -------------------------------------------------
qrrechnung-definitions: | provide definitions and functions for qrrechnung
                 Author:| Benedikt Trefzer
                 E-mail:| benedikt.trefzer@cirrax.com
                License:| Released under the LaTeX Project Public License v1.3c or later
                    See:| http://www.latex-project.org/lppl.txt
\endpreamble
\postamble

Copyright (C) 2020 by Benedikt Trefzer, Cirrax GmbH <benedikt.trefzer@cirrax.com>

This work may be distributed and/or modified under the
conditions of the LaTeX Project Public License (LPPL), either
version 1.3c of this license or (at your option) any later
version.  The latest version of this license is in the file:

http://www.latex-project.org/lppl.txt

This work is "maintained" (as per LPPL maintenance status) by
(not set).

This work consists of the file qrrechnung-translations.dtx

\endpostamble
\usedir{tex/latex/qrrechnung-definitions}
\generate{
  \file{\jobname.sty}{\from{\jobname.dtx}{package}}
}
%</install>
%<install>\endbatchfile
%<*internal>
\usedir{source/latex/qrrechnung-definitions}
\generate{
  \file{\jobname.ins}{\from{\jobname.dtx}{install}}
}
\nopreamble\nopostamble
\usedir{doc/latex/qrrechnung-definitions}
\ifx\fmtname\nameofplainTeX
  \expandafter\endbatchfile
\else
  \expandafter\endgroup
\fi
%</internal>
% \fi
%
% \iffalse
%<*driver>
\ProvidesFile{qrrechnung-definitions.dtx}
%</driver>
%<package>\NeedsTeXFormat{LaTeX2e}[1999/12/01]
%<package>\ProvidesPackage{qrrechnung-definitions}
%<*package>
    [2020/02/06 v1.00 A new LaTeX package]
%</package>
%<*driver>
\documentclass{ltxdoc}
\usepackage[a4paper,margin=25mm,left=50mm,nohead]{geometry}
\usepackage[numbered]{hypdoc}
\usepackage{\jobname}
\EnableCrossrefs
\CodelineIndex
\RecordChanges
\begin{document}
  \DocInput{\jobname.dtx}
\end{document}
%</driver>
% \fi
%
% \GetFileInfo{\jobname.dtx}
% \DoNotIndex{\newcommand,\newenvironment}
%
%\title{\textsf{qrrechnung-definitions} --- A new LaTeX package\thanks{This file
%   describes version \fileversion, last revised \filedate.}
%}
%\author{Benedikt Trefzer, Cirrax GmbH\thanks{E-mail: benedikt.trefzer@cirrax.com}}
%\date{Released \filedate}
%
%\maketitle
%
%\changes{v1.00}{2020/02/06}{First public release}
%
% \begin{abstract}
% Provides definitions and functions for texts used in qrrechnung.sty.
% \end{abstract}
%
% \section{Usage}
%
% only used in qrrechnung.
%
%
%\StopEventually{^^A
%  \PrintChanges
%  \PrintIndex
%}
%
% \section{Implementation}
%
%    \begin{macrocode}
%<*package>
%    \end{macrocode}

%    \begin{macrocode}
\RequirePackage{xstring}
\RequirePackage{ifthen}
\RequirePackage{tikz}
%    \end{macrocode}

% \begin{macro}{\qrr@font}
% select the font to use for text
%    \begin{macrocode}
\newcommand{\qrr@font}{
% \fontfamily{phv}\selectfont
  \fontspec{Liberation Sans}
  %\fontspec{Arial}
}
%    \end{macrocode}
% \end{macro}

% \begin{macro}{\qrr@formatTitle}
% format of the title
%    \begin{macrocode}
\newcommand{\qrr@formatTitle}[1] {
  \qrr@font\fontsize{11pt}{11pt}\selectfont
  \begin{bfseries}
  #1%
  \end{bfseries}
}
%    \end{macrocode}
% \end{macro}

% \begin{macro}{\qrr@formatSubTitleE}
% format of the subtitle on the receipt
%    \begin{macrocode}
\newcommand{\qrr@formatSubTitleE}[1] {
  \qrr@font\fontsize{6pt}{6pt}\selectfont   % minimum 6pt
  \begin{bfseries}
  #1%
  \end{bfseries}
}
%    \end{macrocode}
% \end{macro}

% \begin{macro}{\qrr@formatSubTitleZ}
% format of the subtitle on the payment part
%    \begin{macrocode}
\newcommand{\qrr@formatSubTitleZ}[1] {
  \qrr@font\fontsize{8pt}{8pt}\selectfont   % minimum 6pt
  \begin{bfseries}
  #1%
  \end{bfseries}
}
%    \end{macrocode}
% \end{macro}

% \begin{macro}{\qrr@formatAnnahmestelle}
% format string
%    \begin{macrocode}
\newcommand{\qrr@formatAnnahmestelle}[1] {
  \qrr@font\fontsize{6pt}{6pt}\selectfont    % minimum 2pt bigger than formatSubTitel
  \begin{bfseries}
  #1%
  \end{bfseries}
}
%    \end{macrocode}
% \end{macro}

% \begin{macro}{\qrr@formatWeiterInformationen}
%    \begin{macrocode}
\newcommand{\qrr@formatWeitereInformationen}[1] {
  \qrr@font\fontsize{7pt}{7pt}\selectfont    
  #1%
}
%    \end{macrocode}
% \end{macro}

% \begin{macro}{\qrr@formatTextE}
%    \begin{macrocode}
\newcommand{\qrr@formatTextE}[1] {
  \qrr@font\fontsize{8pt}{8pt}\selectfont    % minimum 2pt bigger than formatSubTitel
  #1%
}
%    \end{macrocode}
% \end{macro}

% \begin{macro}{\qrr@formatTextZ}
%    \begin{macrocode}
\newcommand{\qrr@formatTextZ}[1] {
  \qrr@font\fontsize{10pt}{10pt}\selectfont    % minimum 2pt bigger than formatSubTitel
  #1%
}
%    \end{macrocode}
% \end{macro}

% \begin{macro}{\qrr@box}
% print a box with corners marks(used if Zahlungspflichtiger or Betrag is not set)
%    \begin{macrocode}
\newcommand{\qrr@box}[2]{
   % #1 width
   % #2 height
   \begin{tikzpicture}
     \draw (0mm,3mm)           -- (0mm,0mm) -- (3mm,0mm); %bottom left
     \draw (0mm,#2 - 3mm)      -- (0mm, #2) -- (3mm,#2);  %top left
     \draw (#1 - 3mm,0mm)          -- (#1, 0mm) -- (#1,3mm); %bottom right
     \draw (#1 - 3mm,#2) -- (#1, #2)  -- (#1,#2-3mm); %top right
   \end{tikzpicture}
}
%    \end{macrocode}
% \end{macro}

% \begin{macro}{\qrr@printifE}
% Macro to only print the subtitle if there is some text (on receipt)
%    \begin{macrocode}
\newcommand{\qrr@printifE}[2]{
    \ifthenelse{\isundefined{#2}}{}{
    \qrr@formatSubTitleE{#1}\\
    \qrr@formatTextE{#2}\\
  }
}
%    \end{macrocode}
% \end{macro}

% \begin{macro}{\qrr@printifZ}
% Macro to only print the subtitle if there is some text (on payment part)
%    \begin{macrocode}
\newcommand{\qrr@printifZ}[2]{
    \ifthenelse{\isundefined{#2}}{}{
    \qrr@formatSubTitleZ{#1}\\
    \qrr@formatTextZ{#2}\\
  }
}
%    \end{macrocode}
% \end{macro}

% \begin{macro}{\qrr@padding}
% Macro to make groups of strings eg. to properly display ESR referencenumber.
% Starts on the right and inserts a delimiter (usually a space).
% Parameters: number of chars to group together; string; char to use for padding;
%    \begin{macrocode}
\newcounter{qrrCpadding}
\def\qrr@padding#1#2#3{
   \StrLen{#2}[\len]
   \setcounter{qrrCpadding}{\numexpr \len}
   \StrSplit{#2}{0}{\pb}{\pa}
   \loop
     \addtocounter{qrrCpadding}{- #1}
     \StrSplit{\pa#3\pb}{ \value{qrrCpadding} }{\pa}{\pb}
   \ifnum\value{qrrCpadding} > 0
   \repeat
   \pb
}
%    \end{macrocode}
% \end{macro}

% \begin{macro}{\qrr@paddingscor}
% Macro to make groups of strings eg. to properly display SCOR referencenumber.
% Starts on the left and inserts a space as delimiter every 4 chars.
% Parameters: string to pad
%    \begin{macrocode}
\newcounter{qrscorpadding}
\def\qrr@paddingscor#1{
   \setcounter{qrscorpadding}{4}
   \StrLen{#1}[\len]
   \StrSplit{#1}{\numexpr \len}{\pa}{\pb}
   \loop
     \StrSplit{\pa{\,}\pb}{ \value{qrscorpadding} }{\pa}{\pb}
     \StrLen{\pa{\,}\pb}[\len]
     \addtocounter{qrscorpadding}{5}
   \ifnum\value{qrscorpadding} <  \numexpr \len
   \repeat
   \pa\,\pb
}
%    \end{macrocode}
% \end{macro}

% \begin{macro}{\qrr@paddingAmount}
%    \begin{macrocode}
\def\qrr@paddingAmount#1{
   \StrLen{#1}[\len]
   \StrSplit{#1}{\numexpr \len - 6}{\fr}{\rp}
   \qrr@padding{3}{\fr}{\,}\rp
}
%    \end{macrocode}
% \end{macro}


% \begin{macro}{\qrr@padqrref}
% Macro add leading zeros to qrreference to have 27 characters.
%    \begin{macrocode}
\newcommand{\qrr@padqrref}[1]{ % #1 = refnumber
 \StrLen{#1}[\len]
 \ifnum\len<27%
   \qrr@padqrref{0#1}
 \else 
   \def\RefPadded{#1}%
 \fi
}

%    \end{macrocode}
% \end{macro}

% \begin{macro}{\qrr@checksum}
% Macro to calculate the checksum of a number modulo 10 recursive.
% Output is the number plus the checksum.
%    \begin{macrocode}
\newcounter{checksumX}
\newcounter{checksumC}
\newcounter{checksumZ}
\newcommand{\qrr@checksum}[1]{
  \StrLen{#1}[\reflen]
  \setcounter{checksumX}{0}
  \setcounter{checksumC}{0}
   \loop
     \addtocounter{checksumX}{1}
     \StrChar{#1}{\value{checksumX}}[\checksumZZ]
     \setcounter{checksumZ}{\checksumZZ + \value{checksumC}}

     \IfStrEq{0} {\the\numexpr\value{checksumZ}}{ \setcounter{checksumC}{0} }{}
     \IfStrEq{10}{\the\numexpr\value{checksumZ}}{ \setcounter{checksumC}{0} }{}
     \IfStrEq{1} {\the\numexpr\value{checksumZ}}{ \setcounter{checksumC}{9} }{}
     \IfStrEq{11}{\the\numexpr\value{checksumZ}}{ \setcounter{checksumC}{9} }{}
     \IfStrEq{2} {\the\numexpr\value{checksumZ}}{ \setcounter{checksumC}{4} }{}
     \IfStrEq{12}{\the\numexpr\value{checksumZ}}{ \setcounter{checksumC}{4} }{}
     \IfStrEq{3} {\the\numexpr\value{checksumZ}}{ \setcounter{checksumC}{6} }{}
     \IfStrEq{13}{\the\numexpr\value{checksumZ}}{ \setcounter{checksumC}{6} }{}
     \IfStrEq{4} {\the\numexpr\value{checksumZ}}{ \setcounter{checksumC}{8} }{}
     \IfStrEq{14}{\the\numexpr\value{checksumZ}}{ \setcounter{checksumC}{8} }{}
     \IfStrEq{5} {\the\numexpr\value{checksumZ}}{ \setcounter{checksumC}{2} }{}
     \IfStrEq{15}{\the\numexpr\value{checksumZ}}{ \setcounter{checksumC}{2} }{}
     \IfStrEq{6} {\the\numexpr\value{checksumZ}}{ \setcounter{checksumC}{7} }{}
     \IfStrEq{16}{\the\numexpr\value{checksumZ}}{ \setcounter{checksumC}{7} }{}
     \IfStrEq{7} {\the\numexpr\value{checksumZ}}{ \setcounter{checksumC}{1} }{}
     \IfStrEq{17}{\the\numexpr\value{checksumZ}}{ \setcounter{checksumC}{1} }{}
     \IfStrEq{8} {\the\numexpr\value{checksumZ}}{ \setcounter{checksumC}{3} }{}
     \IfStrEq{18}{\the\numexpr\value{checksumZ}}{ \setcounter{checksumC}{3} }{}
     \IfStrEq{9} {\the\numexpr\value{checksumZ}}{ \setcounter{checksumC}{5} }{}

     \ifnum\value{checksumX} < \reflen
       \repeat
   \IfStrEq{0} {\the\numexpr\value{checksumC}}{\def\RefFull{#10}}{}
   \IfStrEq{1} {\the\numexpr\value{checksumC}}{\def\RefFull{#19}}{}
   \IfStrEq{2} {\the\numexpr\value{checksumC}}{\def\RefFull{#18}}{}
   \IfStrEq{3} {\the\numexpr\value{checksumC}}{\def\RefFull{#17}}{}
   \IfStrEq{4} {\the\numexpr\value{checksumC}}{\def\RefFull{#16}}{}
   \IfStrEq{5} {\the\numexpr\value{checksumC}}{\def\RefFull{#15}}{}
   \IfStrEq{6} {\the\numexpr\value{checksumC}}{\def\RefFull{#14}}{}
   \IfStrEq{7} {\the\numexpr\value{checksumC}}{\def\RefFull{#13}}{}
   \IfStrEq{8} {\the\numexpr\value{checksumC}}{\def\RefFull{#12}}{}
   \IfStrEq{9} {\the\numexpr\value{checksumC}}{\def\RefFull{#11}}{}
}
%    \end{macrocode}
% \end{macro}

% \begin{macro}{\qrrechnung@EzahlungsempfaengerQR}
% This is a dummy function for the final receipient, a data group, that is
% currently unused accoring to standard version 2.1
%    \begin{macrocode}
\newcommand{\qrrechnung@EzahlungsempfaengerQR}{
     \?% typ
     \?% name
     \?% adresse1
     \?% adresse2
     \?% plz
     \?% ort
     \?% land
}
%    \end{macrocode}
% \end{macro}

% \begin{macro}{\qrr@qrcodetext}
% This is the actual text, that is merged into the qrcode.
%    \begin{macrocode}
\newcommand{\qrr@qrcodetext}{
  SPC\?                                  % QRtype
  0200\?                                 % Version
  1\?                                    % CodingType (UTF-8 Latin Characters)
  \qrrechnung@ibanQR                         % IBAN or QR-IBAN
  \qrrechnung@zahlungsempfaengerQR             % Empfänger
  \qrrechnung@EzahlungsempfaengerQR             % endgültigrt Empfänger
  \qrrechnung@betragQR                         % Betrag
  \qrrechnung@waehrungQR                       % Waehrung
  \qrrechnung@zahlungspflichtigerQR            % zahlungpflichtiger
  \qrrechnung@referenzQR                       % Referenznummer
  \qrrechnung@unstrukturierteinfoQR            % unstrukturierte Mitteilung
  EPD\?                                  % Trailer
  \qrrechnung@rechnungsinfoQR            % Rechnungsinformation
  \qrrechnugn@AVerfahrenAQR              % Alternatives verfahren 1
  \qrrechnugn@AVerfahrenBQR              % Alternatives verfahren 2
}
%    \end{macrocode}
% \end{macro}

\endinput
%    \begin{macrocode}
%</package>
%    \end{macrocode}
%\Finale


